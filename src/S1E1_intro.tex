\documentclass[10pt]{beamer}
\usetheme{metropolis}

\usepackage{amsmath, booktabs, fontawesome5, natbib, subfigure, xcolor}

\usepackage[font=small,skip=0pt]{caption}

\usepackage{pgfplots}
\usepgfplotslibrary{dateplot}

\usepackage{xspace}
\newcommand{\themename}{\textbf{\textsc{metropolis}}\xspace}

\usepackage{graphicx}
\graphicspath{{../imgs/}}

\usepackage[T1]{fontenc}
\usepackage[french]{babel}

\usepackage{appendixnumberbeamer}

\usepackage{tikz}
\usetikzlibrary{shapes.geometric, arrows}
\tikzstyle{rect} = [rectangle, rounded corners, minimum width=2cm, minimum height=1.5cm, text centered, draw=black, fill=black!30]
\tikzstyle{squa} = [square, rounded corners, minimum width=1.25cm, minimum height=1.25cm, text centered, draw=black, fill=black!30]
\tikzstyle{elli} = [ellipse, minimum width=1.25cm, minimum height=1cm, text centered, draw=black, fill=black!30]
\tikzstyle{circ} = [circle, minimum width=1cm, minimum height=1cm, text centered, draw=black, fill=black!30]
\tikzstyle{arrow} = [thick,->,>=stealth]
\tikzstyle{drrow} = [thick,<->,>=stealth]
\tikzstyle{dline} = [dashed, ->, >=stealth]
\tikzstyle{dotted} = [densely dotted, ->, >=stealth]

\def\firstcircle{(90:1.75cm) circle (2.5cm)}
\def\secondcircle{(210:1.75cm) circle (2.5cm)}
\def\thirdcircle{(330:1.75cm) circle (2.5cm)}
\tikzset{fontscale/.style = {font=\relsize{#1}}}

\titlegraphic{\hfill\includegraphics[height=1.25cm]{~/Documents/scpobx/logo.pdf}}
\title{Méthodes des sciences sociales}
\subtitle{Séance 1 : Introduction générale}
\author{Mickael Temporão}
\institute{\faEnvelope~$m.temporao@sciencespobordeaux.fr$\\\faTwitter~$@mickaeltemporao$}
\date{}


\begin{document}

\maketitle

\begin{frame}{Méthodes des sciences sociales}
    \begin{block}{Ordre du jour}
        \begin{itemize}
            \item[\faBook] Théorie : Fonctionnement du cours
            \item[\faFlask] Pratique : Groupes et ordre de passage
            \item[\faLaptopCode] Technique : Github \faGithubSquare
        \end{itemize}
    \end{block}
\end{frame}

\section{\faBook~Théorie }

\begin{frame}{Fonctionnement du cours}
    \begin{center}
    \includegraphics[height=6cm, trim=0 0.5 3.7cm .5, clip]{fine.jpg}
    \end{center}
\end{frame}

\begin{frame}{Fonctionnement du cours}
    \includegraphics[height=6cm]{fine.jpg}
\end{frame}

\begin{frame}{Fonctionnement du cours}

    \begin{block}{Objectifs}
        \begin{itemize}
            \item<2->[1.] Compétences de recherche empirique
            \item<3->[2.] Compréhension conceptuelle des approches
            \item<4->[3.] Utilisation d'outils de recherche \textit{open source}
        \end{itemize}
    \end{block}

    \onslide<5->{
    \begin{block}{Évaluation}
        \begin{itemize}
            \item<6-> 30\% : Participation
            \item<7-> 30\% : Travaux Pratiques
            \item<8-> 40\% : Rapport de recherche
        \end{itemize}
    \end{block}
}

\end{frame}


\begin{frame}{Fonctionnement du cours}

        \begin{block}{Deux semestres \onslide<2->{\faArrowRight~2 Parties}}
            \begin{itemize}
                \item<3->[1.] Méthodes Qualitatvies
                \item<4->[2.] Méthodes Quantitatives
            \end{itemize}
        \end{block}

    \onslide<5->{
    \begin{block}{Chaque semestre  \onslide<5->{\faArrowRight~3 Sections}}
    \small
    \begin{center}
        \begin{tikzpicture}[node distance=1.5cm]
        \onslide<6->{
            \node (design) [elli, xshift=-8cm, fill=none] {1. Projet};
        }
        \onslide<7->{
            \node (collection) [elli, xshift=2cm, right of=design, fill=none] {2. Collecte};
            \draw [arrow] (design) -- (collection);
        }
        \onslide<8->{
            \node (analysis) [elli, xshift=2cm, right of=collection, fill=none] {3. Analyse};
            \draw [arrow] (collection) -- (analysis);
        }
        \onslide<9->{
            \node (tp1) [below of=design, fill=none] {\large{\textbf{TP1}}};
            \draw [arrow] (design) -- (tp1);
        }
        \onslide<10->{
            \node (tp2) [below of=collection, fill=none] {\large{\textbf{TP2}}};
            \draw [arrow] (collection) -- (tp2);
        }
        \onslide<11->{
            \node (rapport) [below of=analysis, fill=none] {\large{\textbf{Paper}}};
            \draw [arrow] (analysis) -- (rapport);
        }
        \end{tikzpicture}
    \end{center}
    \end{block}
}

\end{frame}


\begin{frame}{Fonctionnement du cours}



\begin{block}{Séances ce semestre}
    \small
\only<1-5>{
\begin{center}
    \begin{tabular}{ l l c c l }
\hline
        \textbf{Groupe A}	& \textbf{Groupe B}	& Section	    & Séance	& Thème \\
        \hline
        2020/09/18	& 2020/09/25	& Projet & 1	    & Introduction \\
        2020/10/02	& 2020/10/09	& Projet & 2   	& Projet de recherche \\
        2020/10/16	& 2020/10/23	& Collecte  & 3   	& Type de données \\
        2020/11/06	& 2020/11/13	& Collecte  & 4   	& Biais et inférence \\
        2020/11/20	& 2020/11/27	& Analyse   & 5   	& Analyse de contenu \\
        2020/12/04	& 2020/12/11	& Analyse   & 6   	& Rapport recherche
    \end{tabular}
\end{center}
}

\only<6->{
\begin{center}
    \begin{tabular}{ l l c c c }
\hline
        \textbf{Groupe A}	& \textbf{Groupe B}	& Section	    & Séance	& Remise \\

        \hline
        2020/09/18	& 2020/09/25	& Design	& 1	    & \\
        2020/10/02	& 2020/10/09	& Design	& 2   	& \\
        2020/10/16	& 2020/10/23	& Collecte  & 3   	& \textbf{\only<6>{\color{red}}TP1} \\
        2020/11/06	& 2020/11/13	& Collecte  & 4   	& \\
        2020/11/20	& 2020/11/27	& Analyse   & 5   	& \textbf{\only<6>{\color{red}}TP2} \\
        2020/12/04	& 2020/12/11	& Analyse   & 6   	& \\
        \onslide<7>{\textbf{\color{red}2021/01/11}	& \textbf{\color{red}2021/01/18}	& --- & --- & \textbf{\color{red}Rapport Final}}
    \end{tabular}
\end{center}
}

\end{block}
    \onslide<2->{
    \begin{block}{Déroulement d'une séance}
    \begin{center}
        \begin{tikzpicture}[node distance=1.5cm]
        \onslide<3->{
            \node (pres) [rect, xshift=-8cm] {\faBook~Théorie};
        }
        \onslide<4->{
            \node (acti) [rect, xshift=2cm, right of=pres] {\faFlask~Pratique};
            \draw [arrow] (pres) -- (acti);
        }
        \onslide<5->{
            \node (disc) [rect, xshift=2cm, right of=acti] {\faLaptopCode~Technique};
            \draw [arrow] (acti) -- (disc);
        }
        \end{tikzpicture}
    \end{center}
    \end{block}
    }
\end{frame}


\section{\faFlask~Pratique}
\begin{frame}{\faFlask~Groupes et ordre}
        \begin{itemize}
            \item Création des groupes
            \item Définition de l'ordre de passage
        \end{itemize}
\end{frame}

\section{\faLaptopCode~Technique}

\begin{frame}{\faLaptopCode~Premiers pas sur Github}
        \begin{itemize}
            \item Création d'un compte Github \faGithubSquare
            \item Création de son premier "repo"
        \end{itemize}
\end{frame}

\begin{frame}[plain]
    \begin{center}
        \includegraphics[width=11cm]{fine.jpg}
    \end{center}
\end{frame}

\end{document}
